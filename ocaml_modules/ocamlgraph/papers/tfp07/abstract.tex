\documentclass[12pt]{article}

\begin{document}

\title{\bf Designing a Generic Graph Library \\ using ML Functors}

\author
    {Sylvain Conchon$^1$ \quad
     Jean-Christophe Filli\^atre$^1$ \quad
     Julien Signoles$^2$}

\date{$^1$ LRI, Univ Paris-Sud, CNRS, Orsay F-91405\\
INRIA Futurs, ProVal, Orsay F-91893\\
  \texttt{\{conchon,filliatr\}@lri.fr} \\[0.5em]
  $^2$ CEA-LIST, Laboratoire S\^uret\'e des Logiciels \\
  \texttt{Julien.Signoles@cea.fr}}

\maketitle
\thispagestyle{empty}

\begin{abstract}
  This paper details the design and implementation of \textsc{Ocamlgraph}, a
  highly generic graph library for the programming language \textsc{Ocaml}.
  This library features a large set of graph data
  structures---directed or undirected, with or without labels on
  vertices and edges, as persistent or mutable data structures,
  etc.---and a large set of graph algorithms.  Algorithms are written
  independently from graph data structures, which allows combining
  user data structure (resp. algorithm) with \textsc{Ocamlgraph} algorithm
  (resp. data structure).  Genericity is obtained through massive use
  of the \textsc{Ocaml} module system and its functions, the so-called
  \emph{functors}.
\end{abstract}

\end{document}

%%% Local Variables: 
%%% mode: latex
%%% TeX-master: t
%%% End: 
