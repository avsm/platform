\documentclass[a4paper,10.5pt]{article}

\usepackage[utf8]{inputenc}
\usepackage{fullpage,url}
\usepackage{listings}
\usepackage{amssymb,amsmath,amsthm}

\title{Internship report at LRI}
\author{Yuto Takei \\ The University of Tokyo }

\begin{document}

\maketitle

\begin{abstract}

This document describes my experience of 5 week internship at the
Laboratoire de Recherche en Informatique (LRI) at Universit\'e
Paris-Sud. My main objective for this internship was to find the
research topic for my Master's project and learn how to conduct it. It
was also expected to broaden my network for prolonging academic
communication.

I was dispatched to ProVal Team at LRI, whose members investigate the
methods to prove the correctness of the program code. My assignments
were implementation of the negative-cycle detection algorithms and
formal verification of \textsc{Bellman-Ford} algorithm under the
supervision of Dr.\ Jean-Christophe Filli\^atre. In order to handle
them, I utilized some tools developed by ProVal Team such as
\emph{ocamlgraph} and \emph{Why3}.  With gaining much support and
instructions from him, I successfully integrated the implementation
and additional by-product to the existing software library and proved
some characteristics of the algorithm to be correct. Also as a result,
I acquired some of the fundamental skill for conducting the research
through taking communications with various researchers.

In the former half of the document, my objectives and results are
explained in terms of research. The section contains the background
motivation and importance of the work as well. In the latter half, my
experiences during the internship are explained in detail. They
include the reception, supervision and social communication with other
researchers.

\end{abstract}

\newpage

\section{Objective and Results}

\paragraph{Motivation and Objectives}

The main objective for my exchange internship was to find the
appropriate topic for my Master's research and to broaden my network
for wider communication with other researchers. My interests in
computer science lay in deep understanding and contribution to the
topics related to both programming languages and computation
theory. It is because the study on these is essential and directly
beneficial in order to put the algorithmic research into practice. At
the same time, I had an interest as well in methods to process graph
data structure efficiently. By combining these interests, I decided to
investigate the graph processing system written in a functional
programming language.

Since those languages have higher ability to let the compilers
optimize the program for execution, I thought it is efficient by
nature to construct discrete algorithms by using them. Among them,
OCaml was most flexible with wider choices to handle complicated data
structure; it provides the interface to operate both persistent and
imperative data. Additionally it is a strong advantage that OCaml is
developed by INRIA and has a close connection with research ground.

By examining noted graph processing systems in OCaml, I found
\emph{ocamlgraph} project, which was ongoing at ProVal Team in
Laboratoire de Recherche en Informatique (LRI) at Universit\'e
Paris-Sud. The project was appealing for me also because LRI has many
researchers from CNRS and INRIA.  Thinking about the diversity of the
laboratory, it was the ideal project to fulfill my objective.

In asking for the acceptance approval, I contacted Prof.\ Christine
Paulin-Mohring, the leader of ProVal Team at LRI and she kindly
introduced Dr.\ Jean-Christophe Filli\^atre, one of the developers of
\emph{ocamlgraph}. He accepted my internship request and suggested me
some assignments:

\vspace*{-4pt}\begin{itemize}\itemsep-3pt
\item Implementation of one or more negative-cycle detection
  algorithms\cite{cg99negcycle} in \emph{ocamlgraph} library
\item Proving the correctness of the implemented algorithms by using
  \emph{Why3} platform and theorem provers (e.g. \emph{Alt-Ergo} as an
  automated prover and \emph{Coq} for an interactive one)
\end{itemize}\vspace*{-4pt}

Fortunately, these assignments completely matched to my interest and
objective. The internship was thus finally settled.

\paragraph{Importance of the Topics}

The assignments shown earlier contain important challenges.

Regarding the first assignment, \emph{ocamlgraph} \cite{conchon07tfp}
is a software library for OCaml language that provides several
implementations to represent a graph and discrete algorithms to
perform on them. The implementation can be chosen in accordance with
the user's program. It is also possible to define a new data structure
optimized for specific purpose.

The implementation of negative-cycle detection algorithm is important
on \emph{ocamlgraph} because the algorithm is essential as a
pre-performed operation before the others to let them work properly.
It has wide range of applications not only in discrete algorithms, but
also in the real world such as currency arbitrage.

Besides, as mentioned in \cite{conchon07tfp}, the enrichment of
\emph{ocamlgraph} should be promoted because that it is the only
successful graph library to provide the wider choices of graph
representations than any other without making the codes unmanageable.
Others either failed in supporting several representations or
algorithms, or/and lost the generality of the algorithm to perform.

Regarding the second assignment, \emph{Why3} \cite{boogie11why3} is a
software verification platform that enables one to describe the
algorithm properties by first-order logic. Same as \emph{ocamlgraph},
it is developed by ProVal Team. The importance of the assignment is of
two-fold.

First, in this field of study, there existed many proof works on
discrete algorithms. However they required either total involvement of
human in proof session or manual division of original goal into
smaller propositions, resulting in the cost enlargement. By enjoying
one of the most important advantages in \emph{Why3}, the ability to
prove the goal by combining several different theorem provers to
expand the availability of partial proofs, it is possible to largely
promote the automation of the entire proof. Therefore, it is worth to
accomplish the proof on the discrete algorithms by \emph{Why3}.

Second, at the end of the project, it was also planned to integrate
the achievement of the proof into the standard libraries in
\emph{Why3} platform. By accomplishing this, logical description of a
graph data, axioms and theorems on graph characteristics, and related
functions and predicates become reusable in other proof sessions. It
largely contributes to the other works on proving discrete algorithms
mostly automatically.

\paragraph{Project Results}

Here I explain the overview of my result from the research point of
view. As for the achievement in communication with other researchers
as noted as one of the objectives above, it will be discussed in the
next section.

First, I implemented \textsc{Bellman-Ford} algorithm taken from
\cite{cor01algo} as one of the fundamental algorithms of
negative-cycle detection algorithm. By integrating the implementation
within \emph{ocamlgraph} library, as much attention was paid not to
lose the generality to handle any arbitrarily graph
representations. Also considering the practical usage, a simple
optimization was done to reduce the number of loop iterations. In the
end, I tested it comprehensively to work correctly for any input.

Additionally as the advanced topic, I also designed and added the
incremental graph builder for \emph{ocamlgraph}, which is available as
a new graph representation and ensures it to be free of negative
cycles at any point. However on the other hand, it does not improve
the worst-case time complexity compared to the trivial method and I
found out it is required to survey the results of dynamic graph
researches.

Second, the correctness proof of the implemented algorithm was half
completed. I found there involves following four desired properties,
and first half of them was proved to be correct:

\begin{enumerate}\itemsep-3pt
\item The algorithm always terminates eventually.
\item Non-reachable vertices from the source vertex have infinity
  length after the termination.
\item Other vertices must have the length of the shortest path from
  the source after the termination, if no negative length cycle exists
  in the graph.
\item If and only if there is at least one negative-cycle in the
  graph, the algorithm reports the existence.
\end{enumerate}

Although not all properties were proved, I learned the basic
techniques for formal verification and gained some observations
throughout this assignment.

\begin{itemize}\itemsep-3pt
\item It is essential to provide the correct specification of the
  algorithm as precondition and postcondition. It should be concise
  without redundancy to keep the whole proof manageable by automated
  provers, but also should be exhaustive to cover all properties.
\item It is sometimes effective to use ghost codes, i.e., the
  redundant assertions when the automated provers cannot infer strong
  enough premises. They are not part of the specification, but it is
  inserted for proving to provide the strong enough hypothesis.
\item The pigeon whole principal on graph path is in high demand for
  proving path-related discrete algorithms. It is expressed as the
  statement that ``If there exists a path between given two vertices,
  there always exists the path between them with less steps than the
  number of all vertices in the graph.''
\end{itemize}

Since I could not finish the proof, I was not able to integrate the
result for \emph{Why3} standard library yet. Therefore, continuous
effort will be made to finish the proof before that. In addition, it
is to be considered to define a simple path as the path between two
vertices without any loop. It was one of the experimental intermediate
observations during the work that the use of simple path may make it
possible to bypass the proof of pigeonhole principle. Because it can
be inductively defined easily, I can expect simpler proof if it works
successful.

As a whole, I would like to continue making the effort to complete
this proof work and publish the result at some conference.

\section{Experiences}

As mentioned in the introduction, Dr.\ Jean-Christophe Filli\^atre
kindly supervised me for all the work during the internship. In this
section, I describe my experience at LRI informally apart from the
research result.

\paragraph{Project and Assignments}

On the very first day of my arrival at LRI, Dr.\ Filli\^atre clearly
showed my first task and instructed the way to start. The first task
was the implementation of \textsc{Bellman-Ford} algorithm in
\emph{ocamlgraph} library, which I think was very appropriate as the
first assignment. It helped me to understand the algorithm with detail
and to measure my skill to handle the later proof problem. As a
result, it also made me familiar with \emph{ocamlgraph} library
itself.

In the end of the first week, I started to make the incremental graph
builder. By using the knowledge I acquired by handling the first
assignment, I was able to deal with this problem as advanced topic on
detecting negative length cycle. Unfortunately I lack of time to refer
the literature from the previous work by others and so I could not
make the efficient implementation in terms of time and space
complexity. Yet still this task let me notice the field of dynamic
graph.

When the second week started, he showed the main assignment during my
internship; that is to prove the correctness of the implemented
algorithm by using \emph{Why3} platform. It was difficult and
challenging enough to spend the rest of my internship period to get
some insights about the problem.  Since my interest lay in the
programming languages and computation theory, this assignment best
matched to me. As a result, I was able to devote myself to understand
the background and fundamental techniques to deal with formal
verification as well as the mechanism of \emph{Why3}.

Through the challenge on my assignment, I also gained the opportunity
to learn basics of \emph{Coq}. For example, some parts of the proof
still required the human interaction although most of the proof was
done automatically. In these parts, \emph{Coq} helped me with
understanding inductional propositions and logical leaps that
automated provers could not resolve. Therefore I had to either add
intermediate lemmas and assertions so that they could resolve, or try
to prove the goal by \emph{Coq} on my own. Eventually this specific
experience motivated me to learn the method to prove some propositions
by induction on \emph{Coq}.

Beside the proof work on \emph{Why3}, it was also the important duty
to put the result into the paper and the presentation. Considering to
present and to publish the result at some occasions, he instructed me
to write the article simultaneously with the proof. Also he gave me
the opportunity to give a talk about the result in front of the other
members of the team at the end of the fourth week. I think it was a
really good training for me to gather the result into slides and let
the audience understand the contents.

Finally, there were some works left which requires some additional
effort to solve. I would like to continue working on it to finish
them.

\paragraph{Supervision}

The supervision of Dr.\ Filli\^atre was well organized and easy to
understand yet deep as it motivated me to learn greedily. Also it is
strongly agreed that his guidance was more than conscientious that I
respected his instructions.

He gave me the opportunity to review the work every day or two. In
this review, I reported my daily accomplishments and problems that I
have. I had a feeling that he evaluated the result properly positive
as for the successful part, and he promptly pointed out both the cause
of problems and ways to improve for unfavorable results. That attitude
of him greatly encouraged me to advance the work.

I also thought that he focused on the efficiency to learn new things
and deal with the assignment. For me to try to learn new things, he
showed me several literatures that I should refer as well as letting
me borrow his books. Besides, some tutorials are also given
appropriately when I was in need. For example, while I did not have
any experience in using LaTeX for formatting documents, he explained
the basic techniques on effective usage of it. He also advised me in
the quickest way to adapt with the Ubuntu environment and Emacs. Owing
to it, I now continue to use it efficiently for my work after the
internship.

Finally, he was also outstanding to instruct the way to present the
result. When my draft slides and practice session of the presentation
lacked some consistency among the contents, he illustrated the way to
clearly explain the results and reviewed the amendment for the final
version till it reached to satisfactory level.

As a whole, I deeply appreciated his supervision for giving me the
opportunity to train myself for starting the research work.

\paragraph{Research Activities and Environment}

On every weekday during the internship, I regularly started the work
from half past nine and ended at six sharp in the evening. The total
work hours were 7 per day by taking one and half hour of lunch
break. I spent most of the time at my assigned room for the
work. Without interruption, I was able to concentrate comfortably on
my work.

Regarding the facility, there is a lounge space for every team that
the members can freely discuss anything with having coffee and tea
with some snacks. Many people gathered in the room to talk after
having lunch at noon, which I thought stimulates the interaction
between researchers. I also found that it helped the people to take
some short rest during the work, resulting in the improvement of their
efficiency during the work.

Apart from the daily work, I also had some other opportunities to
cultivate my research ground. One of them was to participate in a
seminar in LRI. It was about bounding the iterations of Linear
Programming given by Prof.\ Shinji Mizuno from Tokyo Institute of
Technology. I thought the atmosphere at the seminar was completely
different from the one we have in my laboratory; audience involves in
the presentation more frankly and discuss what they thought without
any hesitation. Under such an environment, the interaction between the
participants is promoted, resulting in wider communication among
researchers on the topic.

Another opportunity was to have a discussion with Prof.\ Abdel Lisser
regarding my objective. He is the leader of GraphComb Team at LRI and
his research includes optimization methods in terms of combinatorics.
He suggested me to check discrete algorithms which relates to matroids
because some of them may still not be done in formal verification
yet. By checking some related works about them, he recommended me to
find the automated proving tasks which would have academic
significance.

Prof.\ Claude March\' e also kindly gave some time to have a
discussion on the efficient way to prove graph related algorithms,
especially the graph representation method for proof and the proving
the pigeonhole principle about graph paths on \emph{Coq}.

\paragraph{Administrative Procedures}

The administrative tasks at LRI were mostly taken care of by
Dr.\ Emmanuel Waller and Mr.\ Jean-Fran\c cois Baffier. Following are
some examples of them but not limited to them.

\begin{itemize}\itemsep-3pt
\item Public transportation subscriptions
\item Entrance control at the building
\item School cafeteria arrangement
\item Reimbursement of expenses
\item Accommodation consideration
\end{itemize}

\paragraph{Other Communications}

I was able to have miscellaneous discussions with other team members
time to time at lunch and other occasions. Some of them are about the
research interests; others are general topics such as educations. Also
with courtesy of Dr.\ Waller and Dr.\ Filli\^atre, I had opportunities
to visit their places to enjoy the private meal.

\section*{Acknowledgement}

This internship was realized by the overseas dispatch program of IST
and I hereby would like to express my sincere gratitude to all people
involved in it. I would like to express my particular appreciation for
LRI faculty members, especially Dr.\ Filli\^atre for the great
supervision and Dr.\ Waller for the kindest care on administration
procedures. I am thankful for OIR/IST staffs and the program committee
of the University of Tokyo for providing me this opportunity as well.
I received the great support from Mrs.\ Sato on the internship
arrangement.

\bibliographystyle{plain}
\bibliography{./biblio}

\end{document}
