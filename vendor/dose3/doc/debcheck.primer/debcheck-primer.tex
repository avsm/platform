%%%%%%%%%%%%%%%%%%%%%%%%%%%%%%%%%%%%%%%%%%%%%%%%%%%%%%%%%%%%%%%%%%%%%%%%%%
%  Copyright (C) 2010-2012  Pietro Abate <pietro.abate@pps.jussieu.fr>   %
%                           Ralf Treinen <ralf.treinen@pps.jussieu.fr>   %
%                           Unversité Paris-Diderot                      %
%                                                                        %
%  This documentation is free software: you can redistribute it and/or   %
%  modify it under the terms of the GNU General Public License as        %
%  published by the Free Software Foundation, either version 3 of the    %
%  License, or (at your option) any later version.                       %
%%%%%%%%%%%%%%%%%%%%%%%%%%%%%%%%%%%%%%%%%%%%%%%%%%%%%%%%%%%%%%%%%%%%%%%%%%

\documentclass{article}
\usepackage[utf8]{inputenc}
\usepackage{hevea}
\usepackage{alltt}

\newcommand{\debcheck}{dose-debcheck}
\newcommand{\distcheck}{dose-distcheck}

\input{version}

\newenvironment{example}%
  {\par\begin{divstyle}{example}\par\textbf{Example:}}%
  {\end{divstyle}\par}
\newstyle{.example}{border:solid black;background:\#eeddbb;padding:lex}

\title{The Dose-Debcheck Primer}
\author{Pietro Abate, Roberto Di Cosmo, Ralf Treinen, Stefano Zacchiroli}
\date{\today}

\begin{document}

\maketitle

The \debcheck{} tool determines, for a set of package control stanzas,
called \emph{the repository}, whether packages of the repository can
be installed relative to the repository. Typically, the repository is
a \texttt{Packages} file of a Debian suite. The installability check
is by default performed for all package stanzas in the repository, but
may be also be restricted to a subset of these.

This primer applies to version \version{} of \debcheck. 

\tableofcontents

%%%%%%%%%%%%%%%%%%%%%%%%%%%%%%%%%%%%%%%%%%%%%%%%%%%%%%%%%%%%%%%%%%%%%%%%%%
%  Copyright (C) 2010-2012  Pietro Abate <pietro.abate@pps.jussieu.fr>   %
%                           Ralf Treinen <ralf.treinen@pps.jussieu.fr>   %
%                           Unversité Paris-Diderot                      %
%                                                                        %
%  This documentation is free software: you can redistribute it and/or   %
%  modify it under the terms of the GNU General Public License as        %
%  published by the Free Software Foundation, either version 3 of the    %
%  License, or (at your option) any later version.                       %
%%%%%%%%%%%%%%%%%%%%%%%%%%%%%%%%%%%%%%%%%%%%%%%%%%%%%%%%%%%%%%%%%%%%%%%%%%

\section{Input Data: Packages and Repositories}
\label{sec:data}
\subsection{Packages}
\label{sec:packages}
Debian control stanzas are defined in \texttt{deb-control~(5)}. For
\debcheck{} only the following fields are relevant, all others are
ignored:
\begin{description}
\item[Package] giving the package name. \debcheck{} is more liberal as
  to which package names are acceptable, for instance it allows a
  slightly larger character set than the debian policy for
  constituting names. Required.
\item[Version] giving the version of the package. The version must be
  in conformance with the Debian policy. Required.
\item[Architecture] specifying the architectures on which the package
  may be installed. Required.
\item[Multiarch] specifies whether the package may be installed
  simultaneously for different architectures, and whether it may
  satisfy dependencies across architecture boundaries. Values may be
  \texttt{No}, \texttt{Same}, \texttt{Foreign}, or \texttt{Allowed}
  (\cite{ubuntu:multiarch}). Optional, defaults to \texttt{No}.
\item[Depends] is a list of items required for the installation of
  this package. Each item is a package name optionally with a version
  constraint, or a disjunction of these.  Items may also be annotated
  with \texttt{:any}. Optional, defaults to the empty list.
\item[Pre-Depends] are by \debcheck{} treated like Depends.
\item[Conflicts] is a list of package names, possibly with a version
  constraint, that cannot be installed together with the package. Optional,
  defaults to the empty list.
\item[Breaks] are by \debcheck{} treated like Conflicts.
\item[Provides] is a list of names symbolizing functionalities
  realized by the package. They have to be taken into account for
  dependencies and conflicts of other packages, see
  Section~\ref{sec:installability}. Optional, defaults to the empty
  list.
\item[Essential] specifies whether the package must be installed
  (\texttt{yes} or \texttt{no}). Optional, defaults to \texttt{no}.
\end{description}

In particular, \texttt{Recommends} and \texttt{Suggests} are ignored
by \debcheck. Also, \debcheck{} does not check for the presence of
fields that are required by Debian policy but that are not relevant
for the task of \debcheck, like \texttt{Maintainer} or
\texttt{Description}.

Also note that \debcheck{} is slightly more liberal than the Debian
policy in accepting input, and hence cannot be used to check strict
policy conformance of package stanzas.

\subsection{Repositories}
\label{sec:repositories}
A \emph{repository} is a set of package stanzas. This set may be given
to \debcheck{} in form of a single file or as several files, in the
latter case the repository is constituted by all stanzas in all input
files (see Section~\ref{sec:invocation}). \debcheck{} assures that the
repositories has two important properties:

\begin{enumerate}
\item
  We assume that there are no two package stanzas in the repository
  that have the same values of all three fields \texttt{Package},
  \texttt{Version}, and \texttt{Architecture}. Having different
  versions for the same package name is OK, as it is of course OK to
  have two stanzas with different package names and the same version.
  In other words, the \debcheck{} tool uses internally the triple of
  name, architecture and version as an identifier for packages.

  In the following, when we speak of \emph{a package}, we mean a
  precise package stanza that is identified by a name, a version, and
  an architecture, like the package of name \texttt{gcc} in version
  \texttt{4:4.3.2-2} and for architecture \texttt{amd64}. The stanza with name
  \texttt{gcc} and version \texttt{4:4.4.4-2} for architecture
  \texttt{amd64} would constitute a different package.

  If the input contains several stanzas with the same name, version
  and architecture then all but the last such stanza are dropped, and a 
  warning message is issued.

\begin{example} The following input does not constitute a repository:
\begin{verbatim}
Package: abc
Version: 42
Architecture: amd64
Depends: xyz

Package: abc
Version: 42
Architecture: amd64
Depends: pqr
\end{verbatim}
The reason is that the triple $(abc,42,amd64)$ is not
unique. \debcheck{} will warn us that it only accepts the second
stanza and drops the first one from its input:
\begin{verbatim}
(W)Debian: the input contains two packages with the same name, version and architecture (abc,42,amd64). Only the latter will be considered.
\end{verbatim}
\end{example}

\item
  We assume that Multiarch information is consistent: If the
  repository contains packages with the same name and version and
  different architecture then both packages have to agree on the value
  of their \texttt{Multiarch} field.

\end{enumerate}



%%%%%%%%%%%%%%%%%%%%%%%%%%%%%%%%%%%%%%%%%%%%%%%%%%%%%%%%%%%%%%%%%%%%%%%%%%
%  Copyright (C) 2010-2012  Pietro Abate <pietro.abate@pps.jussieu.fr>   %
%                           Ralf Treinen <ralf.treinen@pps.jussieu.fr>   %
%                           Unversité Paris-Diderot                      %
%                                                                        %
%  This documentation is free software: you can redistribute it and/or   %
%  modify it under the terms of the GNU General Public License as        %
%  published by the Free Software Foundation, either version 3 of the    %
%  License, or (at your option) any later version.                       %
%%%%%%%%%%%%%%%%%%%%%%%%%%%%%%%%%%%%%%%%%%%%%%%%%%%%%%%%%%%%%%%%%%%%%%%%%%

\section{Installability}
\label{sec:installability}

\subsection{A Precise Definition}

In order to understand what installability exactly means for us we
need a little bit of theory.  Let $R$ be a repository (see
Section~\ref{sec:data}).  An \emph{$R$-installation set}, or sometimes
simply called an \emph{$R$-installation}, is a subset $I$ of $R$ that
has the following four properties:
\begin{description}
  \item[flatness:] $I$ does not contain two different packages with
    the same name (which then would have different versions or
    architecture), unless the package is marked as
    \texttt{Multiarch=Same}. If package $(p,a,n)$ has
    \texttt{Multiarch=Same} then $I$ just must not contain any package
    with name $p$ and a version different from $n$.
  \item[abundance:] For each package $p$ in $I$, every of its
    dependencies is satisfied by some package $q$ in $I$, either
    directly or through a virtual package in case the dependency does
    not carry a version constraint. 
    \begin{itemize}
    \item If $q$ has a Multiarch value of \texttt{No} or \texttt{Same}
      then the architecture of $q$ must be the same as the
      architecture of $p$.
    \item If $q$ has a Multiarch value of \texttt{Foreign} then the
      architecture of $q$ may be different then the architecture of $p$.
    \item If $q$ has a Multiarch value of \texttt{Allowed} then the
      architecture of $q$ must be the same as the architecture of $p$,
      or the dependency relation must carry the annotation \texttt{:any}.
    \end{itemize}
    In this context, the architecture value \texttt{all} is identified with
    the native architecture \cite{ubuntu:multiarch}.
  \item[peace:] For each package in $I$ and for each item in its list
    of conflicts, no package in $I$ satisfies the description of that
    item.  As an exception, it is allowed that a package in $I$ both
    provides a virtual package and at the same time conflicts with it.
  \item[foundation:] If package $(p,n)\in R$ is essential, then $I$
    must contain a package $(p,m)$ such that $(p,m)$ is essential.
\end{description}
Hence, the notion of an installation captures the idea that a certain
set of packages may be installed together on a machine, following the
semantics of binary package relations according to the Debian Policy.
The foundation requirement expresses that essential packages must be
installed; it is formulated in a way that also caters to the
(extremely rare) case that a package changes its \texttt{Essential}
value between different versions. The foundation property may be
switched off by giving the option \texttt{--deb-ignore-essential}.

\begin{example}
  Let $R$ be the following repository:
\begin{verbatim}
    Package: a
    Version: 1
    Depends: b (>= 2) | v

    Package: a 
    Version: 2
    Depends: c (> 1)

    Package: b
    Version: 1
    Conflicts: d

    Package: c
    Version: 3
    Depends: d
    Conflicts: v

    Package: d
    Version: 5
    Provides: v
    Conflicts: v
\end{verbatim}

The following subsets of $R$ are not $R$-installation sets:
\begin{itemize}
\item The complete set $R$ since it is not flat (it contains two
  different packages with name $a$)
\item The set $\{(a,1), (c,3)\}$ since it not abundant (the dependency
  of $(a,1)$ is not satisfied, nor is the dependency of $(c,3)$).
\item The set $\{(a,2), (c,3), (d,5)\}$ since it is not in peace
  (there is conflict between $(c,3)$ and $(d,5)$ via the virtual package $v$)
\end{itemize}
Examples of $R$-installation sets are
\begin{itemize}
\item The set $\{(d,5)\}$ (self conflicts via virtual packages are ignored)
\item The set $\{(a,1), (b,1)\}$
\item The set $\{(a,1), (d,5)\}$
\end{itemize}
\end{example}

A package $(p,n)$ is said to be \emph{installable} in a repository $R$
if there exists an $R$-installation set $I$ that contains $(p,n)$.

\begin{example}
  In the above example, $(a,1)$ is $R$-installable since it is contained
  in the $R$-installation set $\{(a,1), (d,5) \}$.

  However, $(a,2)$ is not $R$-installable: Any $R$-installation set
  containing $(a,2)$ must also contain $(c,3)$ (since it is the only
  package in $R$ that can satisfy the dependency of $(a,2)$ on $c
  (>1)$, and in the same way it must also contain $(d,5)$. However, this
  destroys the peace as $(c,3)$ and $(d,5)$ are in conflict. Hence, no such
  $R$-installation set can exist.
\end{example}

\subsection{What Installability does Not Mean}

\begin{itemize}
\item Installability in the sense of \debcheck{} only concerns the
  relations between different binary packages expressed in their
  respective control files. It does not mean that a package indeed
  installs cleanly in a particular environment since an installation
  attempt may still fail for different reasons, like failure of a
  maintainer script or attempting to hijack a file owned by another
  already installed package.
\item Installability means theoretical existence of a solution. It
  does not mean that a package manager (like \texttt{aptitude},
  \texttt{apt-get}) actually finds a way to install that package.
  This failure to find a solution may be due to an inherent
  incompleteness of the dependency resolution algorithm employed by
  the package manager, or may be due to user-defined preferences that
  exclude certain solutions.
\end{itemize}

\subsection{Co-installability}
\label{sec:coinstallability}
One also should keep in mind that, even when two packages are
$R$-installable, this does not necessarily mean that both packages can
be installed \emph{together}. A set $P$ of packages is called
$R$-\emph{co-installable} when there exists a single $R$-installation
set extending $P$.

\begin{example}
  Again in the above example, both $(b,1)$ and $(d,5)$ are
  $R$-installable; however they are not $R$-co-installable.
\end{example}

See Section~\ref{sec:tricks} on how co-installability can be encoded.

%%%%%%%%%%%%%%%%%%%%%%%%%%%%%%%%%%%%%%%%%%%%%%%%%%%%%%%%%%%%%%%%%%%%%%%%%%
%  Copyright (C) 2010-2012   Pietro Abate <pietro.abate@pps.jussieu.fr>   %
%                           Ralf Treinen <ralf.treinen@pps.jussieu.fr>   %
%                           Unversité Paris-Diderot                      %
%                                                                        %
%  This documentation is free software: you can redistribute it and/or   %
%  modify it under the terms of the GNU General Public License as        %
%  published by the Free Software Foundation, either version 3 of the    %
%  License, or (at your option) any later version.                       %
%%%%%%%%%%%%%%%%%%%%%%%%%%%%%%%%%%%%%%%%%%%%%%%%%%%%%%%%%%%%%%%%%%%%%%%%%%

\section{Invocation}
\label{sec:invocation}

\subsection{Basic usage}

\debcheck{} accepts several different options, and also arguments.

\begin{alltt}
  \debcheck{} [option] ... [file] ...
\end{alltt}

The package repository is partionend into a \emph{background} and a
\emph{foreground}. The foreground contains the packages we are actually
interested in, the background contains packages that are just available
for satisfying dependencies, but for which we do not care about installability.

All arguments are interpreted as filenames of Packages input files,
the contents of which go into the foreground. If no argument is given
then metadata of foreground packages is read from standard input.  In
addition, one may specify listings of foreground packages with the
option \verb|--fg=<filename>|, and listings of background packages
with the option \verb|--bg=<filename>|. Input from files (but not from
standard input) may be compressed with gzip or bzip2, provided
\debcheck{} was compiled with support for these compression libraries.

The option \texttt{-f} and \texttt{-s} ask for a listing of uninstallable,
resp.\ installable packages. The option \texttt{-e} asks for an explanation
of each reported case. The exact effect of these options will be explained
in Section~\ref{sec:output}.

\begin{example}
We may check whether packages in \textit{non-free} are installable,
where dependencies may be satisfied from \textit{main} or \textit{contrib}:
\begin{verbatim}
dose-distcheck  -f -e \
    --bg=/var/lib/apt/lists/ftp.fr.debian.org_debian_dists_sid_main_binary-amd64_Packages\
    --bg=/var/lib/apt/lists/ftp.fr.debian.org_debian_dists_sid_contrib_binary-amd64_Packages\
    /var/lib/apt/lists/ftp.fr.debian.org_debian_dists_sid_non-free_binary-amd64_Packages
\end{verbatim}
\end{example} 

\subsection{Checking only selected packages}
\label{sec:invocation-background}
The initial distinction between foreground and background packages is
modified when using the \verb|--checkonly| option. This option takes
as value a comma-separated list of package names, possibly qualified
with a version constraint. The effect is that only packages that match
one of these package names are kept in the foreground, all others are
pushed into the background.

\begin{example}
\begin{alltt}
\debcheck{} --checkonly "libc6, 2ping (= 1.2.3-1)" Packages
\end{alltt}
\end{example}

\subsection{Checking for co-installability}
\label{sec:invocation-coinst}
Co-installability of packages can be easily checked with the
\verb|--coinst| option. This option takes as argument a
comma-separated list of packages, each of them possibly with a version
constraint. In that case, \debcheck{} will check whether the packages
specified are co-installable, that is whether it is possible to
install these packages at the same time (see
Section~\ref{sec:coinstallability}).

Note that it is possible that the name of a package, even when
qualified with a version constraint, might be matched by several
packages with different versions. In that case, co-installability will
be checked for \emph{each} combination of real packages that match the
packages specified in the argument of the \verb|--coinst| option.
\begin{example}
  Consider the following repository (architectures are omitted for
  clarity):
\begin{verbatim}
Package: a
Version: 1

Package: a 
Version: 2

Package: a
Version: 3

Package: b
Version: 10

Package: b
Version: 11

...
\end{verbatim}
Executing the command \verb|debcheck --coinst a (>1), b| on this
repository will check co-installability of 4 pairs of packages: there
are two packages that match \verb|a (>1)|, namely package \texttt{a} in
versions 2 and 3, and there are two packages that match \texttt{b}. Hence,
the following four pairs of packages will be checked for co-installability:
\begin{enumerate}
\item (a,2), (b,10)
\item (a,2), (b,11)
\item (a,3), (b,10)
\item (a,3), (b,11)
\end{enumerate}
\end{example}

Mathematically speaking, the set of checked tuples is the Cartesian product
of the denotations of the single package specifications.

\subsection{Changing the Notion of Installability}

Some options affect the notion of installability:
\begin{itemize}
\item \texttt{--deb-ignore-essential} drops the Foundation requirement
  of installation sets (Section~\ref{sec:installability}). In other
  words, it is no longer required that any installation set contains all
  essential packages.
\end{itemize}

Other options concern Multiarch:
\begin{itemize}
\item \texttt{--deb-native-arch=}\textit{a} sets the native
  architecture to the value $a$. Note that the native architecture is
  not necessarily the architecture on which the tool is executed, it
  is just the primary architecture for which we are checking
  installability of packages. In particular, packages with the
  architecture field set to \texttt{all} are interpreted as packages of the
  native architecture \cite{ubuntu:multiarch}.
\item \texttt{--deb-foreign-archs=}$a_1,\ldots,a_n$ sets the foreign
  architectures to the list $a_1,\ldots,a_n$. Packages may only be installed
  when their architecture is the native architecture (including \texttt{all}),
  or one of the foreign architectures.
\end{itemize}


\subsection{Filtering Packages and Multiarch}
Filtering out packages is a different operation than pushing packages
into the background (Section~\ref{sec:invocation-background}): Background
packages are still available to satisfy dependencies, while filtering out a 
package makes it completely invisible.

\begin{itemize}
\item The effect of \texttt{--latest} is to keep only the latest version of any
package.
\end{itemize}


\subsection{Other Options}
Other options controlling the output are explained in detail in
Section~\ref{sec:output}. A complete listing of all options can be found in
the \debcheck(1) manpage.



%%%%%%%%%%%%%%%%%%%%%%%%%%%%%%%%%%%%%%%%%%%%%%%%%%%%%%%%%%%%%%%%%%%%%%%%%%
%  Copyright (C) 2010-2012  Pietro Abate <pietro.abate@pps.jussieu.fr>   %
%                           Ralf Treinen <ralf.treinen@pps.jussieu.fr>   %
%                           Unversité Paris-Diderot                      %
%                                                                        %
%  This documentation is free software: you can redistribute it and/or   %
%  modify it under the terms of the GNU General Public License as        %
%  published by the Free Software Foundation, either version 3 of the    %
%  License, or (at your option) any later version.                       %
%%%%%%%%%%%%%%%%%%%%%%%%%%%%%%%%%%%%%%%%%%%%%%%%%%%%%%%%%%%%%%%%%%%%%%%%%%

\section{Output}
\label{sec:output}
The output of \debcheck{} is in the YAML format, see
Section~\ref{sec:tricks-python} for how to parse the output.

Without any particular options, \debcheck{} just reports some
statistics:
\begin{example}
\begin{verbatim}
% dose-debcheck rep1
background-packages: 0
foreground-packages: 4
total-packages: 4
broken-packages: 1
\end{verbatim}
\end{example}

With the options \texttt{--failures} and \texttt{--successes}, \debcheck{}
reports findings of the requested kind for all packages in the foreground.
These options may be used alone or in combination. In any case, the status
field tells whether the package is found to be installable (value \texttt{ok})
or non-installable (value \texttt{broken}).

\begin{example}
\begin{verbatim}
% dose-debcheck --failures --successes rep1
report:
 -
  package: a
  version: 1
  architecture: amd64
  source: a (= 1)
  status: broken
  
 -
  package: a
  version: 2
  architecture: amd64
  source: a (= 2)
  status: ok
  
 -
  package: b
  version: 1
  architecture: amd64
  source: b (= 1)
  status: ok
  
 -
  package: c
  version: 3
  architecture: amd64
  source: c (= 3)
  status: ok
  
 
background-packages: 0
foreground-packages: 4
total-packages: 4
broken-packages: 1
\end{verbatim}
\end{example}

With an additional \texttt{--explain} option, an explanation is given
with each finding. 

\subsection{Understanding Explanations of Installability}

An explanation of installability simply consists of an
installation set in the sense of Section~\ref{sec:installability}
containing the package in question.

\begin{example}
\begin{verbatim}
% dose-debcheck --explain --successes rep1
report:
 -
  package: a
  version: 2
  architecture: amd64
  source: a (= 2)
  status: ok
  installationset:
   -
    package: c
    version: 3
    architecture: amd64
   -
    package: a
    version: 2
    architecture: amd64
 -
  package: b
  version: 1
  architecture: amd64
  source: b (= 1)
  status: ok
  installationset:
   -
    package: b
    version: 1
    architecture: amd64
\end{verbatim}
\end{example}

An installation set contains all essential packages (see
Section~\ref{sec:installability}), which may blow up the output of
installability. Giving the option \texttt{--deb-ignore-essential} will
avoid this, but will also alter the notion of installability in some
corner cases (for instance, when a package needs a version of an
essential package that is not available in the repository).

\subsection{Understanding Explanations of Non-installability}

Installability of a package is much easier to explain than
non-installability. The reason for this is that in the former case we
just have to give one installation that our tool has found, while in
the latter case we have to explain why \emph{all} possible attempts to
install the package must fail. The first consequence of this
observation is that the explanation in case of non-installability may
consist of several components.

\begin{example}
  Consider the following repository consisting of only two packages:
\begin{verbatim}
Package: a
Version: 1
Depends: b | c

Package: c
Version: 3
Conflicts: a
\end{verbatim}
To explain why package (\texttt{a},1) is not installable we have to
say why all possible alternative ways to satisfy its dependency must
fail:
\begin{itemize}
\item there is no package \texttt{b} in the repository
\item the only version of package \texttt{c} in the repository is in
  conflict with package (\texttt{a},1)
\end{itemize}
\end{example}

There may be several ways to satisfy dependencies due to alternatives
in the dependencies in packages. Alternatives may occur in dependencies
in different forms:
\begin{itemize}
\item explicitly, like in \texttt{Depends: b|c},
\item through dependency on a package that exists in several versions,
\item through dependency on a virtual package which is provided by several
  (possibly versions of) real packages.
\end{itemize}
There is one component in the explanation for every possible way to
choose among these alternatives in the dependencies.

There are only two possible reasons why an attempt to satisfy dependencies
may fail:
\begin{enumerate}
\item dependency on a package that is missing from the repository,
\item dependency on a package that is in conflict with some other package
  we depend on (possibly through a chain of dependencies).
\end{enumerate}
Each component of the explanation is either a missing package, or a conflict. 

\subsubsection{Explanation in Case of a Missing Package}
A component of the explanation that corresponds to the case of a
missing package consist of two stanzas:
\begin{itemize}
\item a \texttt{pkg} stanza that states the package that cannot satisfy
  one of its direct dependencies
\item a \texttt{depchains} stanza containing the dependency chain that
  leads from the package we have found non-installable to the one that
  cannot satisfy its direct dependency.
\end{itemize}
\begin{example}
An explanation might look like this:
\begin{verbatim}
package: libgnuradio-dev
version: 3.2.2.dfsg-1
architecture: all
source: gnuradio (= 3.2.2.dfsg-1)
status: broken
reasons:
   -
    missing:
     pkg:
      package: libgruel0
      version: 3.2.2.dfsg-1+b1
      architecture: amd64
      unsat-dependency: libboost-thread1.40.0 (>= 1.40.0-1)
     depchains:
      -
       depchain:
        -
         package: libgnuradio-dev
         version: 3.2.2.dfsg-1
         Architecture: all
         Depends: libgnuradio (= 3.2.2.dfsg-1)
        -
         package: libgnuradio
         ersion: 3.2.2.dfsg-1
         architecture: all
         depends: libgnuradio-core0
        -
         package: libgnuradio-core0
         version: 3.2.2.dfsg-1+b1
         architecture: amd64
         depends: libgruel0 (= 3.2.2.dfsg-1+b1)
\end{verbatim}
This tells us that \texttt{libgnuradio-dev} in version $3.2.2.dfsg-1$
is not installable, due to the fact that package \texttt{libgruel0}
in version $3.2.2.dfsg-1+b1$ has a dependency
\texttt{libboost-thread1.40.0 (>= 1.40.0-1)} that is not matched by
any package in the repository. The dependency chain tells why package
\texttt{libgnuradio-dev} in the given version might want to install
\texttt{libgruel0}.
\end{example}

The depchains component gives all possible dependency chains (\textit{depchains}, for short) from the root package
(\texttt{libgnuradio-dev} in the above example) to the one where a
direct dependency is not matched by any package (\texttt{libgruel0} in
the example). We do not include the last node in the dependency chain
to avoid a useless repetition.

In general there may be more then one path to reach a certain package
from a given root package, in that case \debcheck{} will unroll all of
them.
\begin{example}
In the following repository, package \texttt{a} is not installable since 
the dependency of package \texttt{d} cannot be satisfied:
\begin{verbatim}
Package: a
Architecture: amd64
Version: 1
Depends: b|c

Package: b
Architecture: amd64
Version: 1
Depends: d

Package: c
Architecture: amd64
Version: 3
Depends: d

Package: d
Architecture: amd64
Version: 42
Depends: x
\end{verbatim}
There are two different ways how \texttt{a} arrives at a dependency on
\texttt{d}. \debcheck{} reports the problem once, but lists the two paths 
from \texttt{a} to \texttt{d}:
\begin{verbatim}
% dose-debcheck -e -f --checkonly a rep1
report:
 -
  package: a
  version: 1
  architecture: amd64
  source: a (= 1)
  status: broken
  reasons:
   -
    missing:
     pkg:
      package: d
      version: 42
      architecture: amd64
      unsat-dependency: x
     depchains:
      -
       depchain:
        -
         package: a
         version: 1
         architecture: amd64
         depends: b | c
        -
         package: b
         version: 1
         architecture: amd64
         depends: d
      -
       depchain:
        -
         package: a
         version: 1
         architecture: amd64
         depends: b | c
        -
         package: c
         version: 3
         architecture: amd64
         depends: d
\end{verbatim}
\end{example}


\subsubsection{Explanation in Case of a Conflict}
The other possible cause of a problem is a conflict. In that case, the
explanation consists of a \texttt{conflict} stanza giving the two
packages that are in direct conflict with each other. Next, we have
two \texttt{depchain} stanzas that lead to the first, resp. the second
of these directly conflicting packages.
\begin{example}
\begin{verbatim}
package: a
  version: 1
  status: broken
  reasons:
   -
    conflict:
     pkg1:
      package: e
      version: 1
     pkg2:
      package: f
      version: 1
     depchain1:
      -
       depchain:
        -
         package: a
         version: 1
         depends: b
        -
         package: b
         version: 1
         depends: e
     depchain2:
      -
       depchain:
        -
         package: a
         version: 1
         depends: d
        -
         package: d
         version: 1
         depends: f
\end{verbatim}
The first part of the \debcheck{} report is as before with details
about the broken package. Since this is a conflict, and all conflicts
are binary, we give the two packages involved in the conflict
first. Packages \texttt{f} and \texttt{e} are in conflict, but they
are not direct dependencies of package \texttt{a} . For this reason,
we output the two paths that from a lead to \texttt{f} or
\texttt{e}. All dependency chains for each conflict are
together. Again, since there might be more than one way from a to
reach the conflicting packages, we can have more then one depchain.
\end{example}
If a conflict occurs between two packages that are both reached
through non-trivial dependency chains then we sometimes speak of a
\emph{deep conflict}.

\subsection{The output in case of co-installability queries}
In case of a co-installability query (with the option
\texttt{--coinst}), the distinction between background and foreground
does no longer make sense since the checks now apply to tuples of packages,
and not to individual packages. As a consequence, the summary looks a bit
different in this case:

\begin{example}
  In the following example, there are 3 different versions of package
  \texttt{aa}, two different versions of package \texttt{bb} and two
  packages with other names, giving rise to 6 pairs of packages to
  check for co-installability. Two pairs out of these 6 are found
  not co-installable:
\begin{verbatim}
% ./debcheck --coinst "aa,bb" coinst.packages 
total-packages: 7
total-tuples: 6
broken-tuples: 2
\end{verbatim}
\end{example}

Listings of co-installable, or non co-installable packages when
requested with the options \texttt{-s}/\texttt{--successes},
resp.\ \texttt{-f}/\texttt{--failures}, are similar as before but now
start on the word \texttt{coinst} instead of \texttt{package}. Explanations
are as before:

\begin{example}
\begin{verbatim}
% ./debcheck --coinst "aa,bb" -s -f -e coinst.simple
report:
 -
  coinst: aa (= 2) , bb (= 11)
  status: ok
  installationset:
   -
    package: aa
    version: 2
    architecture: all
   -
    package: bb
    version: 11
    architecture: all
   -
    package: cc
    version: 31
    architecture: all
 -
  coinst: aa (= 1) , bb (= 11)
  status: broken
  
 reasons:
  -
   conflict:
    pkg1:
     package: aa
     version: 1
     architecture: all
     source: aa (= 1)
     unsat-conflict: cc
    pkg2:
     package: cc
     version: 31
     architecture: all
     source: cc (= 31)
    depchain2:
     
  -
   conflict:
    pkg1:
     package: aa
     version: 1
     architecture: all
     source: aa (= 1)
     unsat-conflict: cc
    pkg2:
     package: cc
     version: 31
     architecture: all
     source: cc (= 31)
    depchain1:
     
    depchain2:
     -
      depchain:
       -
        package: bb
        version: 11
        architecture: all
        depends: cc
  
 
total-packages: 5
total-tuples: 2
broken-tuples: 1
\end{verbatim}
\end{example}

%%%%%%%%%%%%%%%%%%%%%%%%%%%%%%%%%%%%%%%%%%%%%%%%%%%%%%%%%%%%%%%%%%%%%%%%%%
%  Copyright (C) 2010-2012  Pietro Abate <pietro.abate@pps.jussieu.fr>   %
%                           Ralf Treinen <ralf.treinen@pps.jussieu.fr>   %
%                           Unversité Paris-Diderot                      %
%                                                                        %
%  This documentation is free software: you can redistribute it and/or   %
%  modify it under the terms of the GNU General Public License as        %
%  published by the Free Software Foundation, either version 3 of the    %
%  License, or (at your option) any later version.                       %
%%%%%%%%%%%%%%%%%%%%%%%%%%%%%%%%%%%%%%%%%%%%%%%%%%%%%%%%%%%%%%%%%%%%%%%%%%

\section{Exit codes}

Exit codes 0-63 indicate a normal termination of the program, codes
64-127 indicate abnormal termination of the program (such as parse
errors, I/O errors).

In case of normal program termination:
\begin{itemize}
\item exit code 0 indicates that all foreground packages are found
  installable;
\item exit code 1 indicates that at least one foreground package is found
  uninstallable.
\end{itemize}

%%%%%%%%%%%%%%%%%%%%%%%%%%%%%%%%%%%%%%%%%%%%%%%%%%%%%%%%%%%%%%%%%%%%%%%%%%
%  Copyright (C) 2010-2012  Pietro Abate <pietro.abate@pps.jussieu.fr>   %
%                           Ralf Treinen <ralf.treinen@pps.jussieu.fr>   %
%                           Unversité Paris-Diderot                      %
%                                                                        %
%  This documentation is free software: you can redistribute it and/or   %
%  modify it under the terms of the GNU General Public License as        %
%  published by the Free Software Foundation, either version 3 of the    %
%  License, or (at your option) any later version.                       %
%%%%%%%%%%%%%%%%%%%%%%%%%%%%%%%%%%%%%%%%%%%%%%%%%%%%%%%%%%%%%%%%%%%%%%%%%%

\section{Tips and Tricks}
\label{sec:tricks}
\subsection{Encoding checks involving several packages}
\debcheck{} only tests whether any package in the foreground set is
installable. However, sometimes one is interested in knowing whether
several packages are co-installable, that is whether there exists an
installation set that contains all these packages. One might also be
interested in an installation that does \emph{not} contain a certain
package.

This can be encoded by creating a pseudo-package that
represents the query. 

\begin{example}
  We wish to know whether it is possible to install at the same time
  \texttt{a} and \texttt{b}, the latter in some version $\geq 42$, but
  without installing c. We create a pseudo package like this:
\begin{verbatim}
Package: query
Version: 1
Architecture: all
Depends: a, b(>= 42)
Conflicts: c
\end{verbatim}
Then we check for installability of that package with respect to the
repository:
\begin{verbatim}
echo "Package: query\nVersion: 1\nArchitecture: all\nDepends: a, b(>=42)\nConflicts: c" | dose-debcheck --bg=repository
\end{verbatim}
(Beware: This might not do exactly what you want, see below!)
\end{example}

The problem with this encoding is as follows: if we ask \debcheck{}
for installability of some package depending on \texttt{a} then this
dependency can a priori be satisfied by any of the available versions
of package \texttt{a}, or even by some other package that provides
\texttt{a} as a virtual package. Virtual packages can be excluded by
exploiting the fact that, in Debian, virtual packages are not
versioned. As a consequence, any package relation (like Depends)
containing a version constraint can only be matched by a real package,
and not by a virtual package. This means that the dependency on
\texttt{b (>= 42)} in the above example already can only be matched by
a real package. If we also want to restrict dependency on \texttt{a}
to real packages only without knowing its possible versions, then we
may write \texttt{Depends: a (>=0) | a(<0)}.

\begin{example}
  If we wish to know whether it is possible to install at the same
  time some version of package \texttt{a} and some version of package
  \texttt{b}, under the condition that these are real packages and not
  virtual packages, then we may construct the following pseudo-package
  and check its installability:
\begin{verbatim}
Package: query
Version: 1
Architecture: all
Depends: a(>=0) | a(<0), b(>=0) | b(<0)
\end{verbatim}
\end{example}

Note that it is in theory possible, though admittedly quite unlikely,
that a package has a version number smaller than $0$ (example:
$0\sim$).

However, if we have several versions of package \texttt{a} and several
versions of package \texttt{b} then the above pseudo-package is
installable if it is possible to install at the same time \emph{some
  version} of \texttt{a} and \emph{some version} of \texttt{b}. If we
want instead to check co-installability of any combination of versions
of package \texttt{a} with versions of package \texttt{b} then the
\texttt{--coinst} option (see Section~\ref{sec:invocation-coinst}) is
better suited for the task.

\subsection{Parsing \debcheck's output in Python}
\label{sec:tricks-python}
Debcheck's output can be easily parsed from a Python program by using
the YAML parser (needs the Debian package \texttt{python-yaml}).

\begin{example}
  If you have run debcheck with the option \texttt{-f} (and possibly
  with the \texttt{-s} option in addition) you may obtain a report
  containing one non-installable package (name and version) per line
  like this:
  
\begin{verbatim}
import yaml

doc = yaml.load(file('output-of-distcheck', 'r'))
if doc['report'] is not None:
  for p in doc['report']:
    if p['status'] == 'broken':
      print '%s %s is broken' (p['package'], p['version'])
\end{verbatim}
\end{example}

A complete example of a python script that constructs a set of
pseudo-packages, runs \debcheck{} on it, and then processes the output
is given in the directory
\texttt{doc/examples/potential-file-overwrites}.

\subsection{Usage as a test in a shell script}
Exit codes allow for a convenient integration of installation checks
as tests in shell scripts.

\begin{example}
Suppose that you want to check installability of all \verb|.deb| files
in the current directory with respect to the repository
\verb|unstable.packages| before uploading your package described in
\verb|mypackage.changes|:

\begin{verbatim}
find . -name "*.deb" -exec dpkg-deb --info '{}' control \; -exec echo ""\; | \
  dose-debcheck --bg unstable.packages && dput mypackage.changes
\end{verbatim}
\end{example}

%%%%%%%%%%%%%%%%%%%%%%%%%%%%%%%%%%%%%%%%%%%%%%%%%%%%%%%%%%%%%%%%%%%%%%%%%%
%  Copyright (C) 2010-2012  Pietro Abate <pietro.abate@pps.jussieu.fr>   %
%                           Ralf Treinen <ralf.treinen@pps.jussieu.fr>   %
%                           Unversité Paris-Diderot                      %
%                                                                        %
%  This documentation is free software: you can redistribute it and/or   %
%  modify it under the terms of the GNU General Public License as        %
%  published by the Free Software Foundation, either version 3 of the    %
%  License, or (at your option) any later version.                       %
%%%%%%%%%%%%%%%%%%%%%%%%%%%%%%%%%%%%%%%%%%%%%%%%%%%%%%%%%%%%%%%%%%%%%%%%%%

\section{Credits}
\label{sec:credits}

Jérôme Vouillon is the author of the solving engine. He also wrote the
first version of the program (called \textsc{debcheck} and
\textsc{rpmcheck} at that time), which was released in November 2005.

The initial development of this tool was supported by the research
project \emph{Environment for the development and Distribution of
  Open Source software (EDOS)}, funded by the European Commission
under the IST activities of the 6th Framework Programme. Further
development and maintenance of the software, together with new
applications building on top of it, was funded by the research project
\emph{Managing the Complexity of the Open Source Infrastructure
  (Mancoosi)}, funded by the European Commission under the IST
activities of the 7th Framework Programme, grant agreement 214898.

The work on this software was partly performed at
\ahref{http://www.irill.org}{IRILL}, the Center for Research and
Innovation on Free Software.


%%%%%%%%%%%%%%%%%%%%%%%%%%%%%%%%%%%%%%%%%%%%%%%%%%%%%%%%%%%%%%%%%%%%%%%%%%
%  Copyright (C) 2010-2012  Pietro Abate <pietro.abate@pps.jussieu.fr>   %
%                           Ralf Treinen <ralf.treinen@pps.jussieu.fr>   %
%                           Unversité Paris-Diderot                      %
%                                                                        %
%  This documentation is free software: you can redistribute it and/or   %
%  modify it under the terms of the GNU General Public License as        %
%  published by the Free Software Foundation, either version 3 of the    %
%  License, or (at your option) any later version.                       %
%%%%%%%%%%%%%%%%%%%%%%%%%%%%%%%%%%%%%%%%%%%%%%%%%%%%%%%%%%%%%%%%%%%%%%%%%%

\section{Further Reading}
The \debcheck{} tool, the underlying theory and its application, was
described in \cite{edos2006ase}.

The paper \cite{edos-debconf08} gives an overview of the theory, and
explains how \debcheck{} is used for various aspect of quality
assurance in Debian.


Checking the relationships between software components is of course
also possible and useful for other models of software packages than
Debian packages. In fact, the \debcheck{} tool is only one flavor of a
more general tool called \distcheck{} which may perform these checks
as well for RPM packages and Eclipse plugins, and in the future
possibly for even more formats. These formats have many things in
common, and the authors of \debcheck{} are convinced that the right
architecture for tools dealing with logical aspects of packages is a
modular one. Such a modular architecture should be centered around a
common universal format for describing the relationships between
packages. This architecture is described in \cite{mpm-cbse11}.



%%%%%%%%%%%%%%%%%%%%%%%%%%%%%%%%%%%%%%%%%%%%%%%%%%%%%%%%%%%%%%%%%%%%%%%%%%
%  Copyright (C) 2010-2012  Pietro Abate <pietro.abate@pps.jussieu.fr>   %
%                           Ralf Treinen <ralf.treinen@pps.jussieu.fr>   %
%                           Unversité Paris-Diderot                      %
%                                                                        %
%  This documentation is free software: you can redistribute it and/or   %
%  modify it under the terms of the GNU General Public License as        %
%  published by the Free Software Foundation, either version 3 of the    %
%  License, or (at your option) any later version.                       %
%%%%%%%%%%%%%%%%%%%%%%%%%%%%%%%%%%%%%%%%%%%%%%%%%%%%%%%%%%%%%%%%%%%%%%%%%%

\section{Copyright and Licence}
\label{sec:copyright}
Copyright \copyright {} 2010, 2011, 2012
Pietro Abate \verb|<pietro.abate@pps.univ-paris-diderot.fr>|,
Ralf Treinen \verb|<ralf.treinen@pps.univ-paris-diderot.fr>|,
and Université Paris-Diderot, for this documentation.

This documentation is free software: you can redistribute it and/or
modify it under the terms of the GNU General Public License as
published by the Free Software Foundation, either version 3 of the
License, or (at your option) any later version.

The software itself is, of course, free software. You can redistribute
and/or modify \distcheck{} (including \debcheck), as well as the
underlying library called \texttt{dose}, under the terms of the GNU
Lesser General Public License as published by the Free Software
Foundation, either version 3 of the License, or (at your option) any
later version.  A special linking exception to the GNU Lesser General
Public License applies to the library, see the precise licence
information of \texttt{dose} for details.



\bibliographystyle{alpha}
\bibliography{mancoosi}

\end{document}
