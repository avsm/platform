%%%%%%%%%%%%%%%%%%%%%%%%%%%%%%%%%%%%%%%%%%%%%%%%%%%%%%%%%%%%%%%%%%%%%%%%%%
%  Copyright (C) 2010-2012  Pietro Abate <pietro.abate@pps.jussieu.fr>   %
%                           Ralf Treinen <ralf.treinen@pps.jussieu.fr>   %
%                           Unversité Paris-Diderot                      %
%                                                                        %
%  This documentation is free software: you can redistribute it and/or   %
%  modify it under the terms of the GNU General Public License as        %
%  published by the Free Software Foundation, either version 3 of the    %
%  License, or (at your option) any later version.                       %
%%%%%%%%%%%%%%%%%%%%%%%%%%%%%%%%%%%%%%%%%%%%%%%%%%%%%%%%%%%%%%%%%%%%%%%%%%

\section{Further Reading}
The \debcheck{} tool, the underlying theory and its application, was
described in \cite{edos2006ase}.

The paper \cite{edos-debconf08} gives an overview of the theory, and
explains how \debcheck{} is used for various aspect of quality
assurance in Debian.


Checking the relationships between software components is of course
also possible and useful for other models of software packages than
Debian packages. In fact, the \debcheck{} tool is only one flavor of a
more general tool called \distcheck{} which may perform these checks
as well for RPM packages and Eclipse plugins, and in the future
possibly for even more formats. These formats have many things in
common, and the authors of \debcheck{} are convinced that the right
architecture for tools dealing with logical aspects of packages is a
modular one. Such a modular architecture should be centered around a
common universal format for describing the relationships between
packages. This architecture is described in \cite{mpm-cbse11}.


